\documentclass{beamer}

\usepackage{polyglossia}
\usepackage{fontspec}
\usepackage{nameref}
\usepackage{ifthen}
\usepackage{minted}

\usefonttheme{professionalfonts}
\usetheme{Antibes}
\useoutertheme{infolines_foot}
\setbeamercovered{transparent=20}

\usepackage[math-style=ISO,vargreek-shape=unicode]{unicode-math}

\setdefaultlanguage[spelling=modern,babelshorthands=true]{russian}
\setotherlanguage{english}

\defaultfontfeatures{Ligatures={TeX}}
\setmainfont{CMU Serif}
\setsansfont{CMU Sans Serif}
\setmonofont{CMU Typewriter Text}
\setmathfont{Latin Modern Math}
\AtBeginDocument{\renewcommand{\setminus}{\mathbin{\backslash}}}

\makeatletter
\newcommand*{\currentname}{\@currentlabelname}
\makeatother
\def\t{\texttt}

\newcommand{\cimg}[2]{%
	\begin{center}%
		\ifthenelse{\equal{#2}{}}{%
			\includegraphics[width=0.75\linewidth]{#1}
		}{%
			\includegraphics[width=#2\linewidth]{#1}
		}%
	\end{center}%
}

\newmintinline[cinl]{c}{} %\c is defined :(
\newmintinline[cpp]{cpp}{}
\newmintinline[python]{python}{}
\newmintinline[bash]{bash}{}
\newmintinline[make]{make}{}
\setminted{obeytabs,tabsize=4,linenos,autogobble}
\newminted{c}{}
\newminted{cpp}{}
\newminted{python}{}
\newminted{bash}{}
\newminted{make}{}

\title[Война вирусов]{Война вирусов\\Походовая онлайн-игра на платформе Android}
\author{Лапшин Дмитрий, Степанов Всеволод}
\institute{СПб АУ РАН}
\date{Осень 2015}

\begin{document}

\begin{frame}
	\titlepage
\end{frame}

\section{Содержание}

\begin{frame}[t]{\currentname}
	\tableofcontents
\end{frame}

\section{О проекте}
\subsection{Описание}

\begin{frame}[t]{Цель проекта}
	Реализовать онлайн-версию игры
	\href{https://ru.wikipedia.org/wiki/\%D0\%92\%D0\%BE\%D0\%B9\%D0\%BD\%D0\%B0\_\%D0\%B2\%D0\%B8\%D1\%80\%D1\%83\%D1\%81\%D0\%BE\%D0\%B2}{,,Война Вирусов``}
	c возможностью игры с другим игроком из списка знакомых или случайным противником.

	\cimg{01.png}{0.25}

	Мы посчитали, что игра интересная и несложная в реализации, а современной реализации, тем более с возможностью сетевой игры, нет.
\end{frame}

\begin{frame}[t]{Общий план}
	\begin{enumerate}
		\item На телефоне есть должно быть клиентское приложение с возможностью играть против бота (ИИ) или другого игрока.
		\item Также игрок должен иметь возможность настраивать отобажаемое имя и цвет фигур.
		\item Просмотр законченных игр по шагам.
		\item Поддержка одновременной работы на многих устройствах.
		\item Серверная часть обрабатывает сетевые игры.
	\end{enumerate}
\end{frame}
\subsection{Архитектура}
\begin{frame}[t]{Архитектура}
	\cimg{04.png}{0.7}
\end{frame}

\begin{frame}[t]{Логика}
	\begin{enumerate}
		\item Информация об игроке, игре, их результатах. Модель базы данных.
		\item Игровая логика -- состояние игры, возможные ходы для текущего игрока.
		\item Интерфейс игрока (\t{Player}) "--- програмной оболочки дейсвтий участника игре, кто бы тот ни был (бот, сетевой игрок, локальный игрок...).
	\end{enumerate}
\end{frame}

\begin{frame}[t]{Клиент}
	\begin{enumerate}
	\item
		Интерфейс:
		\begin{enumerate}
			\item Отдельные окна меню, настроек, истории.
			\item \t{GameActivityBase} "--- основа окон для отображения игры с полем и управлением.
			\item \t{GameActivity}, \t{GameActivityRelpay} "--- конечные реалицазии окон.
			\item Инфраструктура отображения фигурок.
		\end{enumerate}

	\item
		Взаимодействие с вервисами Google:
		\begin{enumerate}
			\item Google Cloud Messaging "--- сетевое взаимодействие.
			\item Google Sign-In "--- авторизация.
		\end{enumerate}

	\item
		Реализация элементов логики
		\begin{enumerate}
			\item \t{HumanPlayer}, \t{AIPlayer}, \t{ClientNetworkPlayer}.
			\item Обращение к БД от Android (SQLite).
		\end{enumerate}
	\end{enumerate}
\end{frame}

\begin{frame}[t]{Сервер}
	\begin{enumerate}
		\item
			\t{GCMHandler} "--- обработчик соединения с Google Cloud Messaging

		\item
			\t{DatabaseHandlrer} "--- обработчик соединения с БД.

		\item
			Сама база данных на движке MySQL.

		\item
			Конфигурация запуска (systemd unit, ...).

		\item Логика
			\begin{enumerate}
				\item Обработка запросов
				\item \t{ServerNetworkPlayer}.
			\end{enumerate}
	\end{enumerate}
\end{frame}

\section{Реализация}
\subsection{Бот}

\begin{frame}[t]{Бот}
	\begin{enumerate}
	\item
		Эвристические подходы не очень понятны, требуют времени и сил на реализацию и проверку, поэтому стратегия переборная.

	\item
		Проблемы:
		\begin{enumerate}
			\item Большая ветвистость дерева перебора (порядка 30-50 на уровень).
			\item За один ход совершается 3 движения.
			\item Медлительность телефонов.
		\end{enumerate}

	\item
		Итоговый вариант
		\begin{enumerate}
			\item Перебор трех движений (т.е. одного хода).
			\item Играем за противника жадно еще ход (тоже три движения).
			\item Оценочная функция -- линейная комбинация количества контроллируемых клеток и клеток под ударом.
		\end{enumerate}

	\item
		Можно улучшить, но уже неплохо.
	\end{enumerate}
\end{frame}

\subsection{Хранение игр}
\begin{frame}[t]{Хранение игр}
	\begin{enumerate}
		\item Для хранения игр, пользователей, их настроек используется база данных (\t{SQLite} на клиенте, \t{MySQL} на сервере).
		\item Хранится только история ходов и идентификаторы пользователей: по этой информации можно восстановить всю игру.
		\item Приостановленная игра также хранится в БД.
		\item В теории, возможна синхронизация с сервером (\textit{не сделано}).
	\end{enumerate}
\end{frame}

\subsection{Игра по сети}
\begin{frame}[t]{Игра по сети}
	\begin{enumerate}
		\item
			Для взаимодействия клиент-сервер используется Google Cloud Messaging.
			Всё общение между клиентами и сервером происходит через него.

		\item
			Для запуска игры серверу посылается запрос о создании игры или обновлении состояния уже идущей.

		\item
			На сервере запускается игра, информация о ней перысылается клиентам, они присылают ходы.
	\end{enumerate}
\end{frame}

\section{Сложности}
\subsection{Сложности}
\begin{frame}[t]{Сложности}
	\begin{description}
		\item[API Google.]
			К сожалению, местами официальная документация содержала внутренние противоречия и неточности.
			Оффициальные примеры, что ещё хуже, иногда не работали.
			Расследование, что не работает и почему, отняло у нас время разработки конечного функционала.

		\item[Сериализация данных по сети.]
			Для пересылки игры нужно сериализовать много данных, и мы не сразу нашли готовое решение.

		\item[Постоянная сериализация данных между окнами.]
			Архитектура Android даёт мало возможностей для передачи данных в исходном виде.

		\item[Отличия структуры API Android и обычной Java, MySQL и SQLite.]
			Код обращений к сети и БД пришлось переписывать дважды.
	\end{description}
\end{frame}

\subsection{Не успели}
\begin{frame}[t]{Не успели}
	\begin{enumerate}
		\item Сервер обрабатывает только одну игру одновременно.
		\item Синхронизация данных между многими клиентами.
	\end{enumerate}
\end{frame}

\section{Спасибо за внимание}
\begin{frame}[t]{Спасибо за внимание}
	Cсылки:
	\begin{description}
		\item[Репозиторий:]	\url{https://github.com/LDVSOFT/War-Of-Viruses}
		\item[APK:] \url{https://yadi.sk/d/79-nsxSGqMFq}
	\end{description}
\end{frame}
\end{document}

\documentclass{beamer}

\usepackage{polyglossia}
\usepackage{fontspec}
\usepackage{nameref}
\usepackage{ifthen}
\usepackage{minted}

\usefonttheme{professionalfonts}
\usetheme{Antibes}
\useoutertheme{infolines_foot}
\setbeamercovered{transparent=20}

\usepackage[math-style=ISO,vargreek-shape=unicode]{unicode-math}

\setdefaultlanguage[spelling=modern,babelshorthands=true]{russian}
\setotherlanguage{english}

\defaultfontfeatures{Ligatures={TeX}}
\setmainfont{CMU Serif}
\setsansfont{CMU Sans Serif}
\setmonofont{CMU Typewriter Text}  
\setmathfont{Latin Modern Math}
\AtBeginDocument{\renewcommand{\setminus}{\mathbin{\backslash}}}

\makeatletter
\newcommand*{\currentname}{\@currentlabelname}
\makeatother
\def\t{\texttt}

\newcommand{\cimg}[2]{%
	\begin{center}%
		\ifthenelse{\equal{#2}{}}{%
			\includegraphics[width=0.75\linewidth]{#1}
		}{%
			\includegraphics[width=#2\linewidth]{#1}
		}%
	\end{center}%
}

\newmintinline[cinl]{c}{} %\c is defined :(
\newmintinline[cpp]{cpp}{}
\newmintinline[python]{python}{}
\newmintinline[bash]{bash}{}
\newmintinline[make]{make}{}
\setminted{obeytabs,tabsize=4,linenos,autogobble}
\newminted{c}{}
\newminted{cpp}{}
\newminted{python}{}
\newminted{bash}{}
\newminted{make}{}

\title[Война вирусов]{Война вирусов\\Походовая онлайн-игра на платформе Android}
\author{Лапшин Дмитрий, Степанов Всеволод}
\institute{СПб АУ РАН}
\date{Осень 2015}

\begin{document}

\begin{frame}
	\titlepage
\end{frame}

\section{Содержание}

\begin{frame}[t]{\currentname}
	\tableofcontents
\end{frame}

\section{О проекте}

\subsection{Цель проекта}

\begin{frame}[t]{\currentname}
	Реализовать онлайн-версию игры
	\href{https://ru.wikipedia.org/wiki/\%D0\%92\%D0\%BE\%D0\%B9\%D0\%BD\%D0\%B0\_\%D0\%B2\%D0\%B8\%D1\%80\%D1\%83\%D1\%81\%D0\%BE\%D0\%B2}{,,Война Вирусов``}
	c возмодностью игры с другим игроком из списка знакомых или случайным противником.

	Игра проходит на клеточном поле $10 \times 10$, игроки по очереди ставят свои фигуры, захватывая свободные клетки и уничтожая клетки соперника.
	\cimg{01.png}{0.2}
\end{frame}

\subsection{Общий план}

\begin{frame}[t]{\currentname}
	\begin{enumerate}
	\item
		Реализация состоит из клиентской и серверной части.
		Клиент позволяет пользователю играть, в то время как сервер соединяет клиентов и обрабатывает игры.
	\pause
	\item
		Уже прошедшие игры хранятся в базе данных: сервер хранит все, клиент хранит только свои.
	\pause
	\item
		Сервер и клиенты между собой общаются через систему Google Cloud Messaging, для первого подключения используется \t{https}.
	\end{enumerate}
\end{frame}

\end{document}
